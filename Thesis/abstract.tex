% !TeX root = main.tex

\chapter*{Abstract}

\indent Little is known about the state of Arctic sea ice at any given instance in time. The harshness of the Arctic naturally limits the amount of in situ data that can be collected, resulting in gathered data being limited in both location and time. Remote sensing modalities such as satellite Synthetic Aperture Radar (SAR) imaging and laser altimetry help compensate for the lack of data, but suffer from uncertainty because of the inherent indirectness. Furthermore, precise remote sensing modalities tend to be severely limited in spatial and temporal availability, while broad methods are more accessible at the expense of precision. This thesis focuses on the intersection of these two problems and explores the possibility of corroborating remote sensing methods to create a precise, accessible source of data that can be used to examine sea ice at local scale.
\par
% {\let\clearpage\relax \chapter*{Zusammenfassung}}
% Deutsche Zusammenfassung hier. 

