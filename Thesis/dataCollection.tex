\chapter{Data Collection}
\label{sec:Data_Collection}

  
In this section, include research relevant to the gathering of data. Include the different metrics that may be captured and the different options presented from different avenues. Much research was done earlier based on the opportunities and limitations of NASA satellites. Include documentation you captured about CryoSat if pertinent to this section.

It may be necessary to include information on SAR imaging in the introduction, so even the non-expert reader can get a functional understanding of the topic.

It may be possible to include figures and other data from the NSIDC In-Situ data repository and discuss the viability of that data as a 'ground-truth' and demonstrate that according to those studies, sea ice was non-normal in distribution. Acknowledge that the data from those studies spanned many miles, and the distribution may sheerly be by variance of ice thickness across the collection area.

Here is where we'll discuss the selection of ICEYE's Data for it's high resolution imaging, and IceSat-2 for it's high-accuracy laser altimetry. It may be possible to include figures to demonstrate how these satellites work, but it may not be pertinent to the thesis (although it would help with understanding).

\section {Laser Altimetry (IceSat-2)}

\section {SAR Imaging (ICEYE)}

------Default Text ----------

Provide in brief the background information for your work/field keeping in mind that maybe your readers do not have experience with topics your reference or address in your thesis. 

In the second part, provide a review of the state of the art relevant to your thesis. Here you present relevant research that relates to your work. 
