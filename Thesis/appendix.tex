% !TeX root = main.tex

\fakesection{Appendix A: ATL10 Product}\label{appendix:A}
%   \chapter{}\label{appendix:A}
\tiny
  \begin{table}[htbp]
   \begin{adjustwidth}{-4em}{-4em}
   \centering
  \begin{tabularx}{1.33\textwidth} {|X|X|l|}
   \hline
   Field Name & Description & Units\\ 
\hhline{|=|=|=|}
   /freeboard\_segment/latitude & The beam footprint's ground latitude & \\ 
   \hline
   /freeboard\_segment/longitude & The beam footprint's ground longitude & \\ 
   \hline
   /freeboard\_segment/beam\_fb\_height & The beam footprint's freeboard height relative to the ice surface height (i.e. snow depth atop the ice surface) & m \\ 
   \hline
   /freeboard\_segment/beam\_fb\_confidence & The confidence of the calculated freeboard height & [0-1]\\ 
   \hline
   /freeboard\_segment/beam\_fb\_quality\_flag & A flag describing the results of the freeboard estimate & [-1,1,2,3,4,5] \\ 
   \hline
   /freeboard\_segment/beam\_fb\_unc & The uncertainty of the calculated freeboard height & [0-1]\\ 
   \hline
   /freeboard\_segment/beam\_refsurf\_ndx & The index to retrieve information about the sea surface from '/reference\_surface\_section/' fields & \\ 
   \hline
   /freeboard\_segment/height\_segment\_id & The ID to retrieve information about the ice surface from '/freeboard\_segment/heights/' fields & \\ 
   \hline
   /freeboard\_segment/seg\_dist\_x & The ground distance of the footprint's distance from IS-2's last equator crossing & m\\ 
\hhline{|=|=|=|}
   /freeboard\_segment/heights/layer\_flag & A boolean flag of whether ice surface estimate was likely clear or obstructed & [0,1]\\ 
   \hline
   /freeboard\_segment/heights/ice\_conc\_ssmi & The measured ice concentration of the beam's footprint & [0-1] \\ 
   \hline
   /freeboard\_segment/heights/\newline \indent height\_segment\_sigma & The standard deviation of the recorded ice surface height & m\\ 
   \hline
   /freeboard\_segment/heights/\newline \indent height\_segment\_ssh\_flag & The classification of sea ice surface of the footprint & [0,1,2]\\ 
   \hline
   /freeboard\_segment/heights/\newline \indent height\_segment\_rms & The root mean square error of the inferred sea ice height determined from the photon distribution & m\\ 
   \hline
   /freeboard\_segment/heights/\newline \indent height\_segment\_height & The estimated height of the sea ice surface (relative to the tide-free mean sea surface height) & m\\ 
   \hline
   /freeboard\_segment/heights/\newline \indent height\_segment\_confidence & The confidence of the ice height estimate & [0,1]\\
\hhline{|=|=|=|}
   /reference\_surface\_section/beam\_refsurf \newline \indent\_dist\_x & The ground distance of the reference sea surface from IS-2's last equator crossing & m\\ 
   \hline
   /reference\_surface\_section/beam\_refsurf \newline \indent\_interp\_flag & The interpolation flag for the reference sea surface & [-1,0,1,2,3]\\ 
   \hline
   /reference\_surface\_section/beam\_fb\_unc \newline \indent\_refsurf & The uncertainty of the reference sea surface's elevation & [0-1]\\ 
   \hline
   /reference\_surface\_section/beam\_refsurf \newline \indent\_mss & The mean sea surface height across the length of the reference surface scale & m\\ 
   \hline
   /reference\_surface\_section/beam\_refsurf \newline \indent\_height & The mean height of the sea surface, across all 6 beams & m\\ 
   \hline
   /reference\_surface\_section/beam\_refsurf \newline \indent\_sigma & The standard deviation of the sea surface & m\\ 
   \hline
  \end{tabularx}
   \end{adjustwidth}
\end{table}
