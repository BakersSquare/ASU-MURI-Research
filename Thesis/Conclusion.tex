\chapter{Conclusion}
\label{sec:Conclusion}

Although this thesis did not yield a model capable of applying IceSat-2's precise LiDAR onto SAR imaging, it demonstrated the difficulty of acquiring and processing related remote sensing data sets. With substantial effort, satellite products can be coordinated to acquire data at nearly equivalent times and places. In the arctic, drifting ice further complicates data collection because ideal conditions rely on the intersection of satellite tracks with each other, and the steadiness of the sea ice. The captured data set of this thesis is rare in its coincidence of LiDAR and high resolution SAR imagery, but can not be validated by buoy movement to be completely accurate. More work will need to be done in modeling single channel, low resolution images before being able to map small LiDAR footprints onto expansive SAR imaging. Accomplishing this task will drastically improve the modern understanding of arctic sea ice, which has implications both in science and navigation.