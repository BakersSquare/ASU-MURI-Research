\chapter{Conclusion}
\label{sec:Conclusion}

The result of this thesis is a rough pipeline that links elusive data to a intuitive method of better modeling sea ice. IceSat-2's semi-frequent orbit provides data that can, with planning, be corroborated with other data sources to bridge the gap between different remote sensing methods. The results of the experimentation do not suggest the model is capable of deducing sea-ice thickness from mere SAR imaging.

\section{Limitations}
While intuitive, the fundamental approach to this problem lies on a set of assumptions that may not always be true. At the root of the deduction of sea-ice thickness using IceSat-2 freeboard measurement is the assumption of hydro-static equilibrium at the measured footprint. The equation relies on the densities of snow, ice, and sea water, which are constants that may vary seasonally and geographically. Furthermore, each footprint in this equation is assuming the absence of any other forces acting on the body. This thesis neither considered nor explored the dynamics of sea-ice floes, meaning that the calculated thickness at any given location may differ not only from error, but from a faulty equation derived from an incomplete physical understanding of the observed body. 

\section{Future Work}
In the future, this topic can be further explored by examining different machine learning models and architectures that may be better suited on 2-channel, low-resolution imaging. To aid this, more data should be collected from both IceSat-2 and ICEYE.


